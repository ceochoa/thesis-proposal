\section{Thesis statement}

Consider a \emph{multiset} $\mathcal{M}$ of size $n$ (e.g.,
$\mathcal{M}=\{1,2,3,3,4,5,6,7,8,9\}$ of size $n=10$).
%
% Frequency, Input Structure.
The \emph{multiplicity} of an element $x$ of $\mathcal{M}$ is the
number $m_x$ of occurrences of $x$ in $\mathcal{M}$ (e.g.,
  $m_3=2$). We call the distribution of the multiplicities
of the elements in $\mathcal{M}$ the \emph{input
  structure}\begin{LONG}
  , and denote it by a set of pairs $(x,m_x)$ (e.g., $\{(1,1)$,
  $(2,1)$, $(3,2)$, $(4,1)$, $(5,1)$, $(6,1)$, $(7,1)$, $(8,1)$,
  $(9,1)\}$ in $\mathcal{M}$)\end{LONG}.
%
As early as 1976, Munro and
Spira~\cite{1976-JComp-SortingAndSearchingInMultisets-MunroSpira}
described a variant of the algorithm {\tt{MergeSort}} using counters,
which optimally takes advantage of the input structure
\begin{LONG}
  (i.e., the multiplicities of the distinct elements)
\end{LONG}
when sorting a multiset $\mathcal{M}$ of $n$ elements. Munro and Spira
measure
the ``difficulty'' of the instance in terms of the ``input structure''
by the entropy function
$\mathcal{H}(m_1, \dots, m_\sigma) =
\sum_{i=1}^\sigma{\frac{m_i}{n}}\log{\frac{n}{m_i}}$. The time
complexity of the algorithm is within
$O(n(1 + \mathcal{H}(m_1, \dots, m_\sigma))) \subseteq
O(n(1{+}\log{\sigma})) \subseteq O(n\log{n})$, where $\sigma$ is the
number of distinct elements in $\mathcal{M}$ and
$m_1, \dots, m_\sigma$ are the multiplicities of the $\sigma$ distinct
elements in $\mathcal{M}$ (such that $\sum_{i=1}^\sigma {m_i}=n$),
respectively.

%% Local Input Order, Runs
Any array $\mathcal{A}$ representing a multiset lists its element in
some order, which we call the \emph{input order} and denote by a tuple
(e.g., $\mathcal{A}=(2,3,1,3,7,8,9,4,5,6)$). Maximal sorted subblocks
in $\mathcal{A}$ are a local form of input order and are called
\emph{runs}~\cite{1973-BOOK-TheArtOfComputerProgrammingVol3-Knuth}
(e.g., $\{(2,3)$, $(1,3,7,8,9)$, $(4,5,6)\}$ in $\mathcal{A}$).
%
As early as 1973,
Knuth~\cite{1973-BOOK-TheArtOfComputerProgrammingVol3-Knuth} described
a variant of the algorithm {\tt{MergeSort}} using a prepossessing step
taking linear time to detect \emph{runs} in the array $\mathcal{A}$\begin{LONG}, which he named \texttt{Natural MergeSort}.
  Mannila~\cite{1985-TCom-MeasuresOfPresortednessAndOptimalSortingAlgorithms-Mannila}
  refined the analysis of the \texttt{Natural MergeSort} algorithm to
  yield a time complexity for sorting an array $\mathcal{A}$ of size $n$
  in time within $O(n(1+\log\rho))\subseteq O(n\lg n)$, where $\rho$
  is the number of \emph{runs} in the $\mathcal{A}$\end{LONG}.
Takaoka~\cite{2009-Chapter-PartialSolutionAndEntropy-Takaoka}
described a new sorting algorithm that optimally takes advantage of the distribution of
the sizes of the runs in the array $\mathcal{A}$, which yields a time complexity within
$O(n(1+\mathcal{H}(r_1, \dots, r_{\rho}))) \subseteq
O(n(1{+}\log{\rho})) \subseteq O(n\log{n})$, where $\rho$ is the
number of runs in $\mathcal{A}$ and $r_1, \dots, r_{\rho}$ are the sizes
of the $\rho$ \emph{runs} in $\mathcal{A}$ (such that
$\sum_{i=1}^\rho {r_i}=n$), respectively.
\begin{LONG}
  It is worth noting that in 1997,
  Takaoka~\cite{1997-TR-MinimalMergesort-Takaoka} first described this
  algorithm in a technical report.
\end{LONG}

%% The questions
This suggests the following questions:
\textbf{\begin{enumerate}
  \item Is there a sorting algorithm for multisets which takes the
    best advantage of both its \emph{input order} and its \emph{input
      structure} in a synergistic way, so that it performs as good as
    previously known solutions on all instances, and much better on
    instances where it can take advantage of both at the same time?
  \item Is there a multiselection algorithm and a deferred data structure
    for answering \texttt{rank} and \texttt{select} queries which
    takes the best advantage not only of both of those notions of
    easiness in the input, but additionally also of notions of
    easiness in the queries, such as the \emph{query structure} and
    the \emph{query order}?
  \end{enumerate}
}

%% Our Results
We answer both questions affirmatively: 
%
In the context of \textsc{Sorting}, this improves upon both algorithms from Munro and Spira~\cite{1976-JComp-SortingAndSearchingInMultisets-MunroSpira} and Takaoka~\cite{2009-Chapter-PartialSolutionAndEntropy-Takaoka}. 
%
In the context of \textsc{MultiSelection} and \textsc{Deferred Data
  Structure} for \texttt{rank} and \texttt{select} on
\textsc{Multisets}, this improves upon Barbay et al.'s
results~\cite{2016-JDA-NearOptimalOnlineMultiselectionInInternalAndExternalMemory-BarbayGuptaRaoSorenson}
by adding 3 new measures of difficulty (input order, input structure
and query order) to the single one previously considered (query
structure). Even though the techniques used by our algorithms are
known, the techniques used to refine the analysis of these algorithms
to show that they improve the state of the art are complex.
%
Additionally, 
%
we correct the analysis of the \texttt{Sorted Set Union} algorithm by Demaine et al.~\cite{2000-SODA-AdaptiveSetIntersectionsUnionsAndDifferences-DemaineLopezOrtizMunro}, and
%
we define a simple yet new notion of ``global'' input order, formed by the number of
  preexisting pivot positions in the input (e.g. $(3,2,1,6,5,4)$ has
  one pre-existing pivot position in the middle), not
mentioned in previous
surveys~\cite{1992-ACMCS-ASurveyOfAdaptiveSortingAlgorithms-EstivillCastroWood,1992-ACJ-AnOverviewOfAdaptiveSorting-MoffatPetersson} nor extensions \cite{2013-TCS-OnCompressingPermutationsAndAdaptiveSorting-BarbayNavarro}.

\section{Background}

\subsection{Sorting}

\subsection{Multiselection Problem}

\subsection{Deferred Data Structures}

\subsection{Compressed Data Structures}

\subsection{Maxima}

\subsection{Convex Hull}

\section{Prior Work}

\subsection{Sorting}

\subsection{Multiselection Problem}

\subsection{Deferred Data Structures}

\section{Research Challenges}

\subsection{Compressed Data Structures}

\subsection{Maxima and Convex Hull}

\subsection{Takaoka Problems [List the problems]}

\section{Plan}

\subsection{Paper 1: Synergistic Sorting, Multiselection and Deferred
  Data Structures}

\subsection{Paper 2: Synergistic Compressed Data Structures [RMQ,
  NSV, PSV, permutations, Sequences]}

\subsection{Paper 3: Synergistic Maxima [and Convex Hull] Algorithms}

%%% Local Variables:
%%% mode: latex
%%% TeX-master: "2016-ThesisProposal-Ochoa"
%%% End:
