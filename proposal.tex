\section{Introduction}
\label{sec:intro}

The computational complexity of most problems is studied in the worst
case over instances of fixed size $n$, for $n$ asymptotically tending
to infinity. This approach was refined for NP-difficult problems under
the term ``parameterized
complexity''~\cite{2006-BOOK-ParameterizedComplexityTheory-FlumGrohe},
for polynomial problems under the term ``Adaptive
Algorithms''~\cite{1992-ACMCS-ASurveyOfAdaptiveSortingAlgorithms-EstivillCastroWood,1992-ACJ-AnOverviewOfAdaptiveSorting-MoffatPetersson},
and more simply for data encodings under the term of ``Data
Compression''~\cite{2013-TCS-OnCompressingPermutationsAndAdaptiveSorting-BarbayNavarro},
for a wide range of problems and data types.
%
  %% Many distinct analysis $\implies$ need to classify them (if only
  %% to teach them)
%
Such a variety of results has motivated various tentative to classify
them, in the context of NP-hard problems with a theory of Fixed
Parameter
Tractability~\cite{2006-BOOK-ParameterizedComplexityTheory-FlumGrohe},
and in the context of sorting in the comparison model with a theory of
reduction between
parameters~\cite{1995-DAM-AFrameworkForAdaptiveSorting-PeterssonMoffat}.
%
%% We present a classification

In the context of ``Adaptive Analisys of Algorithms'' the results can
be partitioned into two categories of techniques: those taking
advantage of the \emph{input order} (e.g., disorder measures for
\textsc{Sorting}~\cite{1992-ACJ-AnOverviewOfAdaptiveSorting-MoffatPetersson,1992-ACMCS-ASurveyOfAdaptiveSortingAlgorithms-EstivillCastroWood}),
\textsc{Convex Hull}
algorithms~\cite{2002-SWAT-AdaptiveAlgorithmsForConstructingConvexHullsAndTriangulationsOfPolygonalChains-LevcopoulosLingasMitchell})
and those taking advantage of the \emph{input structure} (e.g., output
sensitive
algorithms~\cite{1986-JCom-TheUltimatePlanarConvexHullAlgorithm-KirkpatrickSeidel},
input order oblivious instance
optimality~\cite{2009-FOCS-InstanceOptimalGeometricAlgorithms-AfshaniBarbayChan}).




  We introduced two other perspectives from which to classify
  algorithms and data structures.
%
  Through the study of the sorting of multisets according to the
  potential ``easiness'' in both the order and the values in the
  multiset, we aimed to introduce a way to classify refined techniques
  of complexity analysis between the ones considering the input order
  and the ones considering the structure in the input; and to show an
  example of the difficulty of combining both into a single hybrid
  algorithmic technique.
%
  Through the study of the online support of \texttt{rank} and
  \texttt{select} queries on multisets according to the potential
  ``easiness'' in both the order and the values in the queries
  themselves (in addition to the potential easiness in the data being
  queried), we aimed to introduce
% 
  %% Categorizing fine grained analysis between data order and
  %% structure
  % - Synergetic solutions
%
  such categorizations. We predict that such analysis techniques will
  take on more importance in the future, along with the growth of the
  block between practical cases and the worst case over instances of
  fixed sizes. Furthermore, we conjecture that synergistic techniques
  taking advantage of more than one ``easiness'' aspect will be of
  practical importance if the block between theoretical analysis and
  practice is to ever be reduced.

\section{Sorting Solutions}
\label{sec:sort}

\subsection{Background}
\label{sec:back}

\subsubsection{Input Order}
\label{sec:order}

\subsubsection{Input Structure}
\label{sec:structure}

\subsection{Synergistic Sorting}
\label{sec:syn-sort}

\subsection{MultiSelection Algorithm}
\label{sec:multiselect}

\subsection{Deferred Data Structures for MultiSets}
\label{sec:dds}

\section{Compressed Data Structures}
\label{sec:compressed}

\section{Maxima Solutions}

\section{Convex Hull Solutions}

%%% Local Variables:
%%% mode: latex
%%% TeX-master: "2016-ThesisProposal-Ochoa"
%%% End:
